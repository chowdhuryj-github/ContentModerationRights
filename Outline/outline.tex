\documentclass[a4paper]{article}
\usepackage{graphicx} 
\usepackage[margin=0.5in]{geometry}
\usepackage{amsmath, booktabs, float, hyperref, graphicx}
\usepackage[style=apa, backend=biber]{biblatex} % Load APA style with biber backend
\addbibresource{references.bib}

\title{Right to Content Moderation}
\author{Salvin Chowdhury}
\date{\today}
\renewcommand{\contentsname}{The Outline}

\begin{document}

\maketitle

\newpage
\tableofcontents
\newpage

\section{Requirements of the Outline \& Abstract}
The purpose of this paper is to craft the outline and abstract with regards to the right to content moderation. We 
look over the outline requirements and the structure of the paper, and then develop an argument for and against the 
thesis statement.

\subsection{Outline Requirements}
The requirements of the paper are to have a title, an abstract that culminates in a thesis statement, a rough outline,
and a list of three properly formatted sources as listed below:
\begin{itemize}
    \item Reading from the class
    \item Entry from the Stanford Encyclopedia of Philosophy
    \item A source of choosing
\end{itemize}

\subsection{What a Good Philosophy Paper Should Look Like}
A good philosophy paper should present a clear thesis and argue for it. The argument needs to be persuasive and 
presented in a well-ordered, logical fashion. We then consider the strongest possible objections to the view and offer
rebuttals. Basic fallacies should be avoided, and appropriate quotations should be used to clarify points.

\subsection{Structure of Paper}
The paper should have the following structure:
\begin{itemize}
    \item An introduction and conclusion
    \item A body that presents the arguments/evidence for the thesis
    \item One or two strong arguments against the thesis
\end{itemize}

\subsection{Sources of Research}
Avoid using Wikipedia, YouTube, and any random website. Examples of good sources are peer-reviewed sources, the Stanford 
Encyclopedia of Philosophy, Google Scholar, JSTOR, Philosopher's Index, and books.

\subsection{Abstract Structure}
The first paragraph should be focused heavily on. Avoid using general introduction statements. Be sure to set a brief
introduction by letting the reader know the topic and why the topic is relevant. The reader should be given a 
play-by-play account of how the essay is going to unfold and include a thesis statement. 

\subsection{Assumptions of the Reader}
Assume that the reader is a jerk and overly hostile. The job is to convince the reader while respecting their 
intelligence. Always give the opposition the best reading and don't assume the reader will do the same. Be sure to 
explain the  concepts fully, and ensure that anyone should be able to read the paper and understand it. 

\subsection{Crafting the Thesis Statement}
The thesis statement should come as the last sentence of the first paragraph. The thesis statement should be true or
false. 

\newpage

\section{Introduction, Title \& Abstract}
In this paper, I discuss about the what the paper is going to be about, the title of the paper, as well as how the
abstract of the paper is going to look like.

\subsection{The Introduction}
This paper discusses in detail about content moderation on the internet. We look at the parties that have been 
historically involved in moderating content, and whether the moderation of the content should be done by humans that
represent an entity that is in charge of the online platform or whether policing of speech should be done by done 
by training an artificial intelligence model that does all the work.

\subsection{Title of the Paper}
As per the introduction, I wanted the title to reflect what the introduction discusses about. Hence, the title of the
paper would be: \\

\textbf{``Who should decide the rules for content moderation and should it be automated using Artificial 
Intelligence?"}

\subsection{The Abstract}
As the debate to who has the right to remove  and moderate content rages on, the different societal views on what may seem morally wrong and 
right greatly differ at the same time. This makes it a great challenge for those in charge, whether it is private 
companies, governmental institutions or even artificial intelligence, on whether they should have the right to hold 
the reins of controlling the free speech that goes on to grow at a exponential rate on the internet. This paper 
explores cases similar to LambdaMOO and decisions made by different groups  on free speech. We argue that it is 
private companies, who should hold the reins of controlling the free speech on the internet  due to a lack of 
political bias, however we argue that it is better to have humans moderating content rather than automating it 
using Artificial Intelligence (AI).

\section{Structure of the Body Paragraphs}
In this section, I discuss about the body of the paper and the kind of content that will be discussed in each body
of the paragraph. The goal is to simply discuss about arguments that support and go against the thesis statement. 
I would like to also expand upon the arguments in detail the demonstrate the strength of these arguments.

\subsection{Paragraph One: The Early Internet \& Moral Philosophies}
In this paragraph, we look at case studies of incidents that occurred within communities on the online internet.
We discuss what decisions were made as a result of such incidents, and who made these decisions and why. We also
create a connection using these case studies with the moral philosophies from Immanuel Kant \& Thomas Hobbes. \\

\noindent The purpose of this paragraph is to demonstrate the real life impacts on human beings when they engage
in online conversations on the internet via a online account. 

\subsubsection{A Rape in Cyberpsace \& Immaneul Kant}
A Rape in Cyberspace is an article by Julian Dibbell. The article discusses about a incident that had occurred on 
LambdaMOO, which is a multi-user dimension program that allows users to create their own characters and perform 
interactions with other characters using just text. One night, a user, ``Bungle'' had decided to take control of two 
other  characters and had performed sadistic actions on them. Although the harm was done virtually, the action had 
translated into physical harm in the form of mental distress for the humans who were in charge of the virtual 
characters. \\

\noindent Later on, knowledge of the incident was known by the entire community, sparking outrage and a demand for 
justice. In a group meeting with all users on the platform, some had advocated for ``Bungle'' be removed while others 
stated that the user had done nothing wrong as the platform had no established rules at the time. While the arguments 
raged on, the moderators had decided at the end to just remove ``Bungle'' from the platform. This article presents how
members of a online community argue about what is justice to them, as well as negative effects of virtual harassment. 
Some of the key takeaways from the article was as follows:

\begin{itemize}
    \item Virtual behavior can have real psychological and physical consequences. The controllers of virtual characters
    have responsibilities for those consequences
    \item The developers and controllers of a virtual environment interface have responsibilities  for the wellbeing
    of the humans that control an online account on a online platform
\end{itemize}

\noindent Here, we can create a connection using these key takeaways with the philosophies of Immanuel Kant with 
regards to the dicussion of perfect and imperfect duty, as well as the duties of online users and moderators to have 
good intention when it comes to interacting with other online users.

\subsubsection{Searching for a Leviathan in Usenet \& Thomas Hobbes}
Usenet is a computer conferencing network where users can send private messages to one another via email. Usenet has no
central authority which monitors access or control, instead, it is done at the site level in the form of communities.
I use this paper to create a connection about online users existing and how they interact on a online platform with
other users with the philosophy of Thomas Hobbes. These are the following connections to be discussed:

\begin{itemize}
    \item How Usenet users on the internet have a ``personae'', which is a personality that a user may embody on a 
    online platform. For example, if a user makes jokes constantly, they may have a ``humorous'' personality
    \item How Usenet users can exert power on other user's using text-based dialogue. For example, using aggressive
    language or attempts at winning a online argument.
\end{itemize}

\noindent Here, we use the philosophy of Thomas Hobbes to demonstrate that online interactions are more than just 
conversations, that they have impacts on the real world. For example, if a online user is persuasive at promoting
left wing ideology and can influence other online users, this will impact the real world in the sense that the 
humans behind the influenced online accounts will express support for such ideologies.

\subsection{Paragraph Two: Who Gets to Moderate?}
After we have established that online interactions on the internet can real consequences, we now look at who should
have the right to moderate over interactions and content on the internet. In this paragraph, we look at arguments
from different papers on who should moderate the internet and why. \\

\noindent The purpose of this paragraph is to demonstrate and present the strength of the arguments as to why and
why not a certain entity should not be allowed to moderate online content.

\subsubsection{Government \& Content Moderation}
The CEO of Cloudflare, Matthew Prince, suggested that the government should be more involved in deciding what speech
is allowed online. But the article argues that it is a bad idea because private companies have been in charge of 
regulating speech on their platforms. If the government had more power over online content, it could threaten free
speech and independence of tech companies. \\

\noindent Tech companies are working on handling harmful content. The paper argues that even though their efforts 
aren't perfect, they are better suited for the job. If the government started controlling what content is allowed, 
it could become a political tool to suppress speech that official don't like. This would be dangerous as it could 
lead to censorship based on political agendas rather than fairness.

\subsubsection{Moderation by the Platform or by the Online Community?}
This paper discusses about the the two different approaches to moderating content on the internet. The first is the
moderating of content by the platform itself, this means the technical administrators who run the platform and have
the power to remove users at will. The second is the moderation of content by online moderators themselves who don't
represent the platform but represents the community they moderate for. \\

The paper details the two perspectives in depth, which is the platforms and policies perspective and the communities
perspectives. The platforms and policies persepctive focuses on content moderation from the perspectives of online
social platforms, and the community perspective focuses on volunteer community moderators who are actually more akin
to community leaders.

\subsection{Paragraph Three: The Argument on Content Moderation by AI}
After we have established the perspectives of different entities being able to moderate content on the internet, we 
now look at if that moderation should be automated by Artificial Intelligence and the papers that support and go 
against AI moderation. \\

\noindent The purpose of this paragraph is to demonstrate and present the strength of the arguments as tho why and
why not the entity in charge should useartificial intelligence should and should not be able to moderate the internet.

\subsubsection{Content Moderation using AI}
Automating content may often be justified due to the sheer size of content on the internet. It is desirable to use AI
due to the immense amount of data , the relentlessness of violations and need to make judgments without human 
moderators making them. \\

\noindent However, the problem is that the AI tools simply compare new posts to previously flagged content. This means
they're great at catching duplicates but not great at identifying new forms of harmful speech. Such AI tools also 
struggle with understanding context, sarcasm and cultural meanings. AI systems also make mistakes that can unfairly 
impact marginalized groups. \\

\noindent As such systems work based on patterns in large amounts of data, they unintentionally 
reinforce existing biases. For example, they may wrongly flag certain communities more often while failing to protect 
them from real harm.

\subsubsection{Fighting Hate Speech, Silencing Drag Queens?}
This article discusses about how AI tools used for moderating content don't fully understand the context, and this may
negatively impact the LGBTQ community. For example, some LGBTQ people, especially drag queens, use words that AI 
systems might label as toxic, however they may be actually playful or empowering. However, because such words are 
used as insults in general conversations, AI tools may assume them as being always offensive.

\section{The Arguments}
In this section, I discuss about the arguments that will be presented in the final version of the paper, as well as 
the counter arguments. These ideas are discussed in the detail to present the strength of these ideas.

\subsection{Modarting of Content by Government}
Here, I explore the arguments as to why governments should be able to moderate content on the internet. I also
explore the counterarguments as to why governments should not have the right to moderate. As a note, I assume that
a government consists of officials who have been elected by their respective constitutuents in a democratic process.
An example of a government would be the United States government.

\subsubsection{Governments Should Not Moderate Content}
Governments should not be able to moderate online content as people who are elected into government represent 
political bias. People who have political biases may have positions on issues, and those positions may seem moral 
to them, a position that may not be shared by groups of inidividuals who oppose such a position. As a result, such 
a bias may be used to silence dissidents who otherwise express dissent against the elected official's views on 
critical issues.

\subsubsection{Governments Shouldn't Moderate Content}
Governments should be able to moderate content as they consist of officials who have been elected in a democratic 
manner. As the responsibility of the government is to uphold the free speech laws that is enshrined in the 
constitution, the people who have the power to uphold rests in the hands of those elected officials. It would not make
sense for a unelected representative to be able to make decisions on content moderation. We justify this using the 
philosophy of Thomas Hobbes, supporting the argument using ideas such as ``Social Contract'' where if inidividuals
should adhere to the constituion in real life, then they should also do this online. Otherwise, it would lead to a 
``State of War''.

\subsection{Moderating of Content by Platorms}
IHere, I explore the arguments as to why platforms should be able to moderate content on the internet. I also explore
the counterarguments as to why platforms should not have the right to moderate.  

\subsubsection{Platforms Should Moderate Content}
Platforms, such as Facebook, should be able to moderate content because if the government had power over online
content, it would threaten free speech and the independence of technology companies. Technology companies are better
suited to handle harmful content than government employees, and have the necessary resources to make it happen. If
the government were able to control what content is allowed, it would become a political tool to suppress free speech.
This would be dangerous as it could lead to censorship.

\subsubsection{Platforms Shouldn't Moderate Content}
Platforms shouldn't be able to moderate content because they may be owned by a board of directors or a chief executive
officer (CEO) who may have political biases. Furthermore, if there are no existing laws that prevent the influence
of such platform executives through donations or bribes, such executives can be easily influenced to control speech
and alter it to make it more favorable to the donor. For example, if a political candidate bribed a social media 
executive to suppress speech that is critical of the candidate and promote speech that favors the candidate, it may
result the candidate winning the election as online users are influenced to believe that the political candidate they
saw on the internet would be a better choice for office than their opponent.

\subsection{Moderating of Content by Volunteer Moderators}
Here, I explore the arguments as to why volunteer moderators should be able to moderate content on the internet. 
I also explore the counterarguments as to why such volunteer moderators should not have the right to moderate. As a 
note, a volunteer moderator is someone who volunteers to moderator and is not a respresentative of the platform
they moderate on. A volunteer moderator is rather a moderator of a online community.

\subsubsection{Volunteer Moderators Should Moderate Content}
Volunteer Moderators should be able to moderate content as the prime motivation behind moderating a community is their
passion for the community they're a part of. Such moderators can't be influenced by any external parties, hence their
motivation to moderate remains true. Furthermore, the survival of a online community depends on the quality of the 
volunteer moderator, as bad moderating could lead to a decrease in interactions within the online community. In other
cases, bad moderating can lead to removal of a volunteer moderator and be replaced by a new one by the will of the 
users of the online community.

\subsubsection{Volunteer Moderators Shouldn't Moderate Content}
Volunteer Moderators shouldn't be able to moderate content because they are vested with the power to censor any user
that may express opinions against them. Their power to censor comes with no consequences, as the user being censored
can no longer access the community they've been removed from to express the injustice that has been comitted against 
them. As a result, good moderation will depend on the mercifulness and maturity of the moderator, something which
can't be guaranteed to be found in all moderators.

\section{Sources}



\end{document}