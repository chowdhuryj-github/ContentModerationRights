\documentclass[a4paper]{article}
\usepackage{graphicx} 
\usepackage[margin=0.5in]{geometry}
\usepackage{amsmath, booktabs, float, hyperref, graphicx}

\title{Outline \& Abstract on Right to Content Moderation}
\author{Salvin Chowdhury}
\date{\today}

\begin{document}

\maketitle

\newpage

\section{Requirements of the Outline \& Abstract}
The purpose of this paper is to craft the outline and abstract with regards to the right to content moderation. We 
look over the outline requirements and the structure of the paper, and then develop an argument for and against the 
thesis statement.

\subsection{Outline Requirements}
The requirements of the paper is to have title, an abstract that culminates in a thesis statement, a rough outline,
and a list of three properly formatted sources as listed below:
\begin{itemize}
    \item Reading from the class
    \item Entry from Stanford Encyclopedia of Philosophy
    \item A Source of Choosing
\end{itemize}

\subsection{What a Good Philosophy Paper Should Look Like}
A good philosophy paper should present a clear thesis and argue for it. The argument needs to be persuasive, and 
presented in a well, ordered logical fashion. We then consider the storngest possible objections to the view and offer
rebuttals. Basic fallacies should be avoided and appropriate quotations should be used to clarify points.

\subsection{Structure of Paper}
The paper should have the following structure:
\begin{itemize}
    \item A introduction and conclusion
    \item A body that presents the arguments / evidence for the thesis
    \item One or two strong argument against the thesis
\end{itemize}

\subsection{Sources of Research}
Avoid using Wikipedia, Youtube and any website. Examples of good sources are peer reviewed sources, Stanford 
Encyclopedia of Philosophy, Google Scholar, JSTOR, Philosopher's Index and Books.

\subsection{Abstract Structure}
The first paragraph should be focused heavily on. Avoid using general introduction statements. Be sure to set a brief
introduction by letting the reader know the topic and why the topic is relevant. The reader should be given a play by
play account of how the essay is going to unfold, and include a thesis statement. 

\subsection{Assumptions of the Reader}
Assume that the reader is a jerk and overly hostile. The job is to convince the reader and respect their intelligence.
Always give the opposition the best reading and don't assume the reader will do the same. Be sure to explain the 
concepts fully, and ensure that anyone should be able to read the paper and understand it. 

\subsection{Crafting the Thesis Statement}
The thesis statement should come as the last sentence of the first paragraph. The thesis statement should be true or
false. 

\newpage

\section{Draft One on The Right to Moderate Content}


\end{document}